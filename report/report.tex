\documentclass[pstricks, 12pt, a4paper]{article}

\usepackage{fontspec}
\usepackage{type1cm}
\usepackage[margin=3cm]{geometry}
\usepackage{amsmath,amsthm,amssymb}
\usepackage{yhmath}
\usepackage{graphicx}
\graphicspath{ {./img/} }
\usepackage[usenames, dvipsnames]{color}
\usepackage[unicode=true, bookmarksdepth=-1]{hyperref}
\usepackage{tabularx}
\usepackage{enumerate}
\usepackage{lastpage}
\usepackage[inline]{enumitem}
\usepackage{esvect}
\usepackage{siunitx}
\usepackage{listings}

\usepackage{xeCJK}
\setCJKmainfont[AutoFakeBold=3,AutoFakeSlant=.4]{Noto Sans TC}
% \setCJKmainfont[AutoFakeBold=3,AutoFakeSlant=.4]{標楷體}

\XeTeXlinebreaklocale "zh"
\XeTeXlinebreakskip = 0pt plus 1pt

\usepackage{fancyhdr}
\usepackage{titlesec}
\usepackage{titling}
\usepackage{lastpage}
\setlength{\headheight}{15pt}
\setlength{\droptitle}{-1.5cm}

\usepackage{multirow} % table content

\usepackage{pst-plot} % drawing
\psset{algebraic}

\usepackage{qtree}

\fancypagestyle{std}{
  \fancyhead[L]{}
  \fancyhead[C]{}
  \fancyhead[R]{計算機概論}
  \fancyfoot[L]{B10902082 林秉軒、B10902028 王\ 勻}
  \fancyfoot[C]{}
  \fancyfoot[R]{Page \thepage\ of \pageref{LastPage}}
  \renewcommand{\headrulewidth}{0.4pt}
  \renewcommand{\footrulewidth}{0.4pt}
}
\pagestyle{std}
\titleformat{\chapter}[block]
  {\thispagestyle{std}\normalfont\huge\bfseries\centering}{\thechapter.}{1em}{\Huge}
\titlespacing*{\chapter}{0pt}{-19pt}{0pt}

\renewcommand{\baselinestretch}{1.3}

\title{Final Project --- Push and Clear}
\author{林秉軒、王\ 勻}
\date{\today}

\newcounter{problem}
\newcounter{subproblem}
\newcounter{solution}

\newcommand{\Problem}[1]{
  {\subsection*{Problem #1}}
}

\newcommand{\Solution}[1]{
  {\subsection*{Solution #1:}}
}

%%%%%%%%%%%%%%%%%%%%%%%%%%%%%%%%%%%%%%%%%%%%%%%%%%%%%%%%%%%%% START
\renewcommand{\bold}[1]{{\boldmath\bfseries#1}}
\renewcommand{\vec}[1]{\vv{#1}}
\renewcommand{\inf}{\infty}
\renewcommand{\Ref}[1]{\textbf{\ref{#1}}}
\newcommand{\thus}{\ \Rightarrow\ }
\newcommand{\then}[2][]{\xrightarrow[#1]{#2}}
\newcommand{\suchthat}{\text{ s.t. }}
\newcommand{\bigbar}{\Bigr|}
\newcommand{\contradict}{\rightarrow\!\leftarrow}
\newcommand{\set}[1]{\left\{#1\right\}}

\newcommand{\block}[3][]{
  \begin{#2}[#1]
    #3
  \end{#2}
}
\makeatletter
\renewcommand*\env@matrix[1][*\c@MaxMatrixCols c]{%
  \hskip -\arraycolsep
  \let\@ifnextchar\new@ifnextchar
  \array{#1}}
\renewcommand{\matrix}[2][*\c@MaxMatrixCols c]{
  \begin{bmatrix}[#1]
    #2
  \end{bmatrix}
}
\newcommand{\resize}[2][1]{\resizebox{#1\textwidth}{!}{#2}}
\newcommand{\inv}[1]{{#1^{-1}}}

\newcommand{\Unit}[2]{\qty[per-mode=symbol, parse-numbers=false]{#1}{#2}}
\newcommand{\Umps}[1]{\Unit{#1}{\metre\per\s}}
\newcommand{\Ucmps}[1]{\Unit{#1}{\metre\per\s}}
\newcommand{\Um}[1]{\Unit{#1}{\metre}}
\newcommand{\Us}[1]{\Unit{#1}{\s}}
\newcommand{\Umpss}[1]{\Unit{#1}{\metre\per\s\squared}}
\newcommand{\Ukg}[1]{\Unit{#1}{\kilogram}}
\newcommand{\UN}[1]{\Unit{#1}{\newton}}
\newcommand{\Uradps}[1]{\Unit{#1}{\radian\per\s}}
\newcommand{\UMeV}[1]{\Unit{#1}{\mega\eV}}
\newcommand{\UC}[1]{\Unit{#1}{\coulomb}}
\newcommand{\UE}[1]{\Unit{#1}{\newton\per\coulomb}}
\newcommand{\UV}[1]{\Unit{#1}{\volt}}
\newcommand{\UW}[1]{\Unit{#1}{\watt}}
\newcommand{\UR}[1]{\Unit{#1}{\ohm}}
\newcommand{\UF}[1]{\Unit{#1}{\farad}}
\newcommand{\UT}[1]{\Unit{#1}{\tesla}}
\newcommand{\UJ}[1]{\Unit{#1}{\joule}}
\newcommand{\UA}[1]{\Unit{#1}{\ampere}}
\newcommand{\UHz}[1]{\Unit{#1}{\hertz}}
\newcommand{\half}{\frac{1}{2}}
\newcommand{\inc}{\uparrow}
\newcommand{\dec}{\downarrow}

\lstdefinestyle{vivid}{
  basicstyle={\small\ttfamily},
  breaklines=true
}
\lstset{style=vivid}
%%%%%%%%%%%%%%%%%%%%%%%%%%%%%%%%%%%%%%%%%%%%%%%%%%%%%%%%%%%%% END
\usepackage{indentfirst}
\parindent 2em
\parskip 1em

\begin{document}

\maketitle

\thispagestyle{std}

% 利用Jack語言實作以上下左右鍵操控的推箱子遊戲

\section{遊戲規則介紹}
在$6\times 6$的盤面中,玩家(實心圓圈)可以透過上下左右移動將整排箱子一次推到底,若六個箱子連成一直排或一橫排即可消除,在原位置留下點數(實心圓點),只要玩家或箱子通過放有點數的格子即可將點數收集成分數。每次玩家操作後都會隨機生成一個新的箱子,其中有一定的機率會產生不可移動的固定箱子(打叉的方格),而當玩家無法移動時,遊戲便會結束。

\section{設計邏輯}
以陣列維護當前的盤面狀態,每次移動時便可快速計算下一個盤面的狀態以及分數,最後再隨機插入新的箱子即可。

\section{程式碼}
  \subsection{亂數模型(Random.jack)}
    我們採用線性同餘法作為亂數模型。在初始化$\texttt{init()}$後,$\texttt{gen()}$會產生一個介於$0\sim 172$的亂數,而$\texttt{next(k)}$則會產生一個介於$0\sim k-1$的亂數(藉由過濾不小於$\lfloor \frac{mod}{k}\rfloor\times k$的亂數使機率分布均等)。
    \begin{lstlisting}[language=Java, frame=single]
class Random {
    static int K, mod;
    function void init() {
        let K = 49;
        let mod = 173;
        return;
    }
    function int gen() {
        let K = (K * 31);
        let K = K - (K / 173 * 173);
        return K;
    }
    function int next(int range) {
        var int up, res;
        let up = mod / range * range;
        let res = up;
        while (~(res < up)) {
            let res = Random.gen();
        }
        let res = res - (res / range * range);
        return res;
    }
}
    \end{lstlisting}

  \subsection{遊戲模型(Game.jack)}
    \subsubsection{fields, constructor}
      儲存並初始化物件在螢幕上的大小排版,盤面狀況、玩家位置、總分和盤面狀態編碼。
      \begin{lstlisting}[language=Java, frame=single]
field int SQUARE_WIDTH, GAP_WIDTH, BOARD_WIDTH;
field int OFFSET_X, OFFSET_Y;

field int N;
field Array board;

field int step;
field int playerX;
field int playerY;
field int totalPoint;

static int EMPTY_CELL, PLAYER, POINT, NORMAL_BLOCK, STABLE_BLOCK;
// 0 - empty cell
// 1 - player
// 2 - a point
// 3 - a normal block
// 4 - a stable block
constructor Game new(int N_) {
    let N = N_;
    let board = Array.new(N * N);
    let SQUARE_WIDTH = 30;
    let GAP_WIDTH = 5;
    let BOARD_WIDTH = (N * SQUARE_WIDTH) + ((N + 1) * GAP_WIDTH);
    let OFFSET_X = (512 / 2) - (BOARD_WIDTH / 2);
    let OFFSET_Y = (256 / 2) - (BOARD_WIDTH / 2);

    let EMPTY_CELL = 0;
    let PLAYER = 1;
    let POINT = 2;
    let NORMAL_BLOCK = 3;
    let STABLE_BLOCK = 4;

    return this;
}
      \end{lstlisting}
    \subsubsection{Basic methods: $\texttt{ended, isInBoard, setBoard}$}
      \noindent $\texttt{ended()}$:透過$\texttt{move(..., false)}$判斷是否四個方位皆不可移動。\\
      $\texttt{isInBoard(x, y)}$:判斷$(x, y)$座標是否在界內。\\
      $\texttt{setBoard(id, x, y, type)}$:將$(x, y)$座標的狀態更改為$\texttt{type}$,並更新此格的畫面顯示。
      \begin{lstlisting}[language=Java, frame=single]
method boolean ended() {
    return ~(move(-1, 0, false) | move(1, 0, false) | move(0, -1, false) | move(0, 1, false));
}

method boolean isInBoard(int x, int y) {
    return (~(x < 0)) & (x < N) & (~(y < 0)) & (y < N);
}

method void setBoard(int id, int x, int y, int type) {
    let board[id] = type;
    let x = (x * (SQUARE_WIDTH + GAP_WIDTH)) + GAP_WIDTH;
    let y = (y * (SQUARE_WIDTH + GAP_WIDTH)) + GAP_WIDTH;
    do Game.drawCell(type, x + OFFSET_X, y + OFFSET_Y, SQUARE_WIDTH);
    return;
}
      \end{lstlisting}
    \subsubsection{$\texttt{move(dx, dy, apply)}$}
      判斷能否往$(dx, dy)$方向移動,若$\texttt{apply = true}$,則同時更新盤面。
      \paragraph{流程}
      \begin{enumerate}
        \item 計算玩家位置到牆壁(或固定箱子)之間的點數、箱子數和空格數。
        \item 藉由點數、空格數總和是否為0判斷此方向是否有空位移動。\\若$\texttt{apply = false}$,可直接回傳。
        \item 將點數累計入分數、箱子靠牆排列,並更新玩家位置。
        \item 呼叫$\texttt{elimination}$函數消除完整的行列,並更新移動步數與分數畫面顯示。
      \end{enumerate}
      \begin{lstlisting}[language=Java, frame=single]
method boolean move(int dx, int dy, boolean apply) {
    var int x, y, id;
    var int tx, ty;
    var int empty, point, normal_block;
    var boolean found, moved;
    let x = playerX + dx;
    let y = playerY + dy;
    let empty = 0;
    let point = 0;
    let normal_block = 0;
    let found = false;
    while ((~found) & isInBoard(x, y)) {
        let id = (x * N) + y;
        if (board[id] = STABLE_BLOCK) {
            let found = true;
        } else {
            if (board[id] = EMPTY_CELL) {
                let empty = empty + 1;
            }
            if (board[id] = POINT) {
                let point = point + 1;
            }
            if (board[id] = NORMAL_BLOCK) {
                let normal_block = normal_block + 1;
            }
            let x = x + dx;
            let y = y + dy;
        }
    }
    if (~apply) {
        return ~((empty + point) = 0);
    }
    // board[x, y] now is out of map or is a stable block
    let tx = x;
    let ty = y;
    let x = tx;
    let y = ty;
    while (~((x = playerX) & (y = playerY))) {
        let x = x - dx;
        let y = y - dy;
        let id = (x * N) + y;
        do setBoard(id, x, y, EMPTY_CELL);
    }
    let x = tx;
    let y = ty;
    while (normal_block > 0) {
        let x = x - dx;
        let y = y - dy;
        let id = (x * N) + y;
        do setBoard(id, x, y, NORMAL_BLOCK);
        let normal_block = normal_block - 1;
    }
    let x = x - dx;
    let y = y - dy;
    let id = (x * N) + y;
    do setBoard(id, x, y, PLAYER);
    let moved = ~((playerX = x) & (playerY = y));
    let playerX = x;
    let playerY = y;

    let totalPoint = totalPoint + point;

    do elimination(true);
    do Output.moveCursor(0, 0);
    do Output.printInt(totalPoint);

    if (moved) {
        let step = step + 1;
    }
    return moved;
}
      \end{lstlisting}
    \subsubsection{$\texttt{elimination(apply)}$}
      將所有完整的行列消除,並回傳是否有任一行列被消除。
      \paragraph{流程}
      \begin{enumerate}
        \item 枚舉所有直行與橫列,並將可消除的行列位置打上標記。
        \item 若$\texttt{apply = false}$,可根據是否有位置被標記判斷有無消除,並直接回傳。
        \item 顯示消除提示文字,並將所有被標記的位置狀態設為點數
      \end{enumerate}
      \begin{lstlisting}[language=Java, frame=single]
method boolean elimination(bool apply) {
    var Array bingo;
    var int i, j, id;
    var boolean flag, hasBingo;
    let bingo = Array.new(N * N);
    let hasBingo = false;
    let id = 0;
    while (id < (N * N)) {
        let bingo[id] = false;
        let id = id + 1;
    }
    let i = 0;
    while (i < N) {
        // vertical
        let j = 0;
        let flag = true;
        while (j < N) {
            let id = (i * N) + j;
            if (~((board[id] = NORMAL_BLOCK) | (board[id] = STABLE_BLOCK))) {
                let flag = false;
            }
            let j = j + 1;
        }
        if (flag) {
            let j = 0;
            while (j < N) {
                let id = (i * N) + j;
                let bingo[id] = true;
                let hasBingo = true;
                let j = j + 1;
            }
        }
        // horizontal
        let j = 0;
        let flag = true;
        while (j < N) {
            let id = (j * N) + i;
            if (~((board[id] = NORMAL_BLOCK) | (board[id] = STABLE_BLOCK))) {
                let flag = false;
            }
            let j = j + 1;
        }
        if (flag) {
            let j = 0;
            while (j < N) {
                let id = (j * N) + i;
                let bingo[id] = true;
                let hasBingo = true;
                let j = j + 1;
            }
        }
        let i = i + 1;
    }
    if (~apply) {
        return hasBingo;
    }
    if (hasBingo) {
        do Output.moveCursor(0, 25);
        do Output.printString("line clear!");
        do Sys.wait(500);
        do Output.moveCursor(0, 25);
        do Output.printString("           ");
    }
    let id = 0;
    while (id < (N * N)) {
        if (bingo[id]) {
            let i = id / N;
            let j = id - (i * N);
            if (board[id] = NORMAL_BLOCK) {
                do setBoard(id, i, j, POINT);
            } else {
                do setBoard(id, i, j, POINT);
            }
        }
        let id = id + 1;
    }
    do bingo.dispose();
    return hasBingo;
}
      \end{lstlisting}
    \subsubsection{$\texttt{generateNewBlock()}$}
      隨機在空白位置生成新箱子,且有$\frac{10\sqrt{steps}}{3}$的機率是固定箱子。為降低自動生成馬上造成消除的可能,生成箱子的位置一旦造成消除,便會重新選擇一個,至多重複三次。而為了避免自動生成固定箱子馬上困住玩家導致遊戲結束,若在選定的位置放置固定箱子後會導致遊戲結束,便以一般的箱子取代之。
      \begin{lstlisting}[language=Java, frame=single]
method void generateNewBlock() {
    var int pos, x, y;
    var int iter, org;
    var boolean flag;
    let iter = 0;
    let flag = true;
    while ((iter < 3) & flag) {
        // try 3 times, if no elimination, break.
        let pos = Random.next(N * N);
        while (~((board[pos] = EMPTY_CELL) | (board[pos] = POINT))) {
            let pos = Random.next(N * N);
        }
        let org = board[pos];
        let x = pos / N;
        let y = pos - (x * N);
        let board[pos] = NORMAL_BLOCK;
        let flag = elimination(false);
        let iter = iter + 1;
        let board[pos] = org;
    }
    if (Random.next(100) < (Math.sqrt(step * 100) / 3)) {
        do setBoard(pos, x, y, STABLE_BLOCK);
        if (ended()) {
            do setBoard(pos, x, y, NORMAL_BLOCK);
        }
    } else {
        do setBoard(pos, x, y, NORMAL_BLOCK);
    }
    do elimination(true);
    return;
}
      \end{lstlisting}
    \subsubsection{$\texttt{generateNewBlock()}$}
      初始化fields,並隨機放置六個一般箱子與玩家位置。
      \begin{lstlisting}[language=Java, frame=single]
method void init() {
    var int i, pos;
    let i = 0;
    while (i < (N * N)) {
        let board[i] = EMPTY_CELL;
        let i = i + 1;
    }
    let i = 0;
    while (i < N) {
        do generateNewBlock();
        let i = i + 1;
    }
    let pos = Random.next(N * N);
    while (~(board[pos] = EMPTY_CELL)) {
        let pos = Random.next(N * N);
    }
    let board[pos] = PLAYER;
    let playerX = pos / N;
    let playerY = pos - (playerX * N);
    let totalPoint = 0;
    let step = 0;
    return;
}
      \end{lstlisting}

  \subsection{主函數(Main.jack)}
    初始化、主迴圈(輸入$\then{}$移動$\then{}$生成$\then{}$顯示)
    \begin{lstlisting}[language=Java, frame=single]
class Main {
    function int readOperation() {
        var int ch;
        let ch = Keyboard.readChar();
        do Output.backSpace();
        if ((ch = 130)) { return 0; }
        if ((ch = 131)) { return 1; }
        if ((ch = 132)) { return 2; }
        if ((ch = 133)) { return 3; }
        return -1;
    }
    function void main() {
        var Game game;
        var int direction, dx, dy;
        var boolean moved;

        let game = Game.new(6);

        do Random.init();
        do game.init();

        do game.draw();
        while (~game.ended()) {
            let direction = Main.readOperation();
            let moved = false;
            if (direction = 0) {
                let dx = -1;
                let dy = 0;
            }
            if (direction = 1) {
                let dx = 0;
                let dy = -1;
            }
            if (direction = 2) {
                let dx = 1;
                let dy = 0;
            }
            if (direction = 3) {
                let dx = 0;
                let dy = 1;
            }
            if (~(direction = -1)) {
                let moved = game.move(dx, dy, true);
            }
            if (moved) {
                do game.generateNewBlock();
            }
        }
        do game.draw();
        do Output.moveCursor(0, 25);
        do Output.printString("game over!");

        do game.dispose();
        return;
    }
}
    \end{lstlisting}

\end{document}
